\chapter{Preparing Binaries} \label{chapter:preparing-binaries}
\paragraph{}
It is likely that you will be able to get a precompiled copy of MCUSim
for your operating system. However, distributing sources of the simulator
should only be expected. This chapter will give you an idea how to prepare,
compile and install a most suitable version of the simulator.

\section{Install dependencies on macOS}
\paragraph{}
There are only two things required for a successful compilation of MCUSim
in general: CMake and ANSI C compiler. Peripheral devices (sensors,
external EEPROM, etc.) are important part of a simulation process and it is
why another component, library to interpret Lua programming language,
is highly recommended.
\emph{brew}\footnote{https://brew.sh/} package manager can be used
to install this one:
\begin{lstlisting}[escapeinside={@}{@}]
	@\$@ brew install cmake lua
\end{lstlisting}
Possible output of an installation process could look like:
\begin{lstlisting}
	==> Pouring cmake-3.9.4_1.sierra.bottle.1.tar.gz
	==> Summary
	/usr/local/Cellar/cmake/3.9.4_1: 2,267 files, 31.7MB
	==> Pouring lua-5.2.4_4.sierra.bottle.1.tar.gz
	==> Summary
	/usr/local/Cellar/lua/5.2.4_4: 145 files, 706KB
\end{lstlisting}

\section{Install dependencies on NetBSD}
\paragraph{}
NetBSD operating system is usually distributed with a very limited number
of preinstalled programs. In this case you will have to configure
the NetBSD Packages Collection, called
\emph{pkgsrc}\footnote{http://www.pkgsrc.org/}. Type in your terminal:
\begin{lstlisting}[escapeinside={@}{@}]
	@\$@ PKG_PATH=@"http://cdn.NetBSD.org/pub/pkgsrc/packages/@
		@\postbreak\$(uname -s)/\$(uname -m)/@
		@\postbreak\$(uname -r|cut -f '1 2' -d.)/All/"@
	@\$@ export PKG_PATH
	@\$@ su
	@\$@ pkg_add -u cmake lua51
\end{lstlisting}
These commands will install or update already installed cross-platform make
program and a library to interpret Lua programming language.

\section{Install dependencies on Windows/Cygwin}
\paragraph{}
It is not necessary to compile MCUSim on a POSIX compatible operating system.
However, AVRSim (part of MCUSim) involves sockets programming to implement
GDB Remote Serial Protocol. Cygwin\footnote{https://www.cygwin.com} is a
recommended way to prepare binaries of the simulator on Windows in this case.

\section{Compilation}
\paragraph{}
You may download sources of the latest version of MCUSim from
\lstinline|http://mcusim.org|. There are two version types of the simulator -
current and stable. First one is updated frequently and contains the latest
features, but cannot be considered as a reliable one and it is not tested
properly. Second one could be outdated, but solid enough for end users.
\paragraph{}
Archive with sources
should look like \lstinline|mcusim-X.X.X.tar.gz| and can be extracted:
\begin{lstlisting}[escapeinside={@}{@}]
	@\$@ tar xvf mcusim-X.X.X.tar.gz
	x mcusim-X.X.X/
	x mcusim-X.X.X/CMakeLists.txt
	x mcusim-X.X.X/COPYING
	x mcusim-X.X.X/README
	x mcusim-X.X.X/avrsim/
	x mcusim-X.X.X/avrsim/avrsim.c
	x mcusim-X.X.X/avrsim/cli.c
	x mcusim-X.X.X/avrsim/decoder.c
	x mcusim-X.X.X/avrsim/gdb_rsp.c
	...
\end{lstlisting}
You may create a new directory and prepare build files for \emph{release}
version of MCUSim:
\begin{lstlisting}[escapeinside={@}{@}]
	@\$@ cd ./mcusim-X.X.X
	@\$@ mkdir out @\&\&@ cd out
	@\$@ cmake -DCMAKE_BUILD_TYPE=Release ..
\end{lstlisting}
Output of the build files generating process can be carefully examined to be
sure that ANSI C compiler, library to interpret Lua 5.1, and POSIX functions
to organize communication over sockets are available. It should contain these
lines:
\begin{lstlisting}
	-- The C compiler identification ...
	-- Check for working C compiler: ...
	-- Check for working C compiler: ... -- works
	-- Detecting C compiler ABI info
	-- Detecting C compiler ABI info - done
	-- Detecting C compile features
	-- Detecting C compile features - done
	...
	-- Release version of MCUSim X.X.X will be built
	-- Found Lua51: ...
	...
	-- Looking for include file netdb.h
	-- Looking for include file netdb.h - found
	-- Looking for include file sys/socket.h
	-- Looking for include file sys/socket.h - found
	-- Looking for include file string.h
	-- Looking for include file string.h - found
	-- Looking for include file fcntl.h
	-- Looking for include file fcntl.h - found
	-- Looking for include file unistd.h
	-- Looking for include file unistd.h - found
	-- Looking for include file errno.h
	-- Looking for include file errno.h - found
	-- Looking for include file poll.h
	-- Looking for include file poll.h - found
	-- Looking for include file netinet/in.h
	-- Looking for include file netinet/in.h - found
	-- Looking for include file netinet/tcp.h
	-- Looking for include file netinet/tcp.h - found
	...
\end{lstlisting}
All build files have been generated and you are ready to compile and install
a release version of the simulator. Type this command in your terminal:
\begin{lstlisting}[escapeinside={@}{@}]
	@\$@ make
	@\$@ make install
\end{lstlisting}
You will likely have to perform the last command as a superuser on many
Unix-like operating systems.
